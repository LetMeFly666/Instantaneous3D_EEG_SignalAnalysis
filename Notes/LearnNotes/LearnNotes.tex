\documentclass[cs4size,a4paper]{ctexart}   
%===数学符号公式===
\usepackage{amsmath}    					% AMS LaTeX宏包
\usepackage[style=1]{mdframed}
\usepackage{amsthm}
\usepackage{amssymb}
\usepackage{bm}                      	% 数学公式中的黑斜体
\usepackage{bbm}
\usepackage{amsfonts}
\usepackage{mathrsfs}                	% 英文花体字 体
\usepackage{bbding,manfnt}    			% 一些图标,如 \dbend
\usepackage{lettrine}                	% 首字下沉,命令\lettrine
\def\attention{\lettrine[lines=2,lraise=0,nindent=0em]{\large\textdbend\hspace{1mm}}{}}
\usepackage{longtable}
\usepackage{enumerate}
\usepackage[toc,page]{appendix}
\usepackage{geometry}         			% 页边距调整
\geometry{top=3.0cm,bottom=2.7cm,left=2.5cm,right=2.5cm}
\usepackage[colorinlistoftodos,prependcaption,textsize=small]{todonotes}
%===公式按章编号===
\numberwithin{equation}{section}
\numberwithin{table}{section}
\numberwithin{figure}{section}
%===基本格式预置===
\usepackage{fancyhdr}
\pagestyle{fancy}
\fancyhf{}  
\fancyhead[C]{\zihao{5}  \kaishu LaTeX软件使用指导}
\fancyfoot[C]{~\zihao{5} \thepage~}
\renewcommand{\headrulewidth}{0.75pt} 
\CTEXsetup[format={\centering\bfseries\zihao{-2}},name={第, 章}]{section}
\CTEXsetup[nameformat={\bfseries\zihao{3}}]{subsection}
\CTEXsetup[nameformat={\bfseries\zihao{4}}]{subsubsection}
%===图形支持宏包===
\usepackage{graphicx}        			% 嵌入png图像
\usepackage{subfigure}
\usepackage{float}
\graphicspath{{figure/}}
\usepackage{color,xcolor}     			% 支持彩色文本、底色、文本框等
\usepackage[colorlinks,linkcolor=blue,anchorcolor=blue,citecolor=blue]{hyperref}
%\usepackage{caption}
\usepackage[ruled,linesnumbered]{algorithm2e}
%\captionsetup{figurewithin=section}
%===源码和流程图===
\usepackage{listings,fontspec}         	% 粘贴源代码
\newfontfamily\monaco{Monaco}
\definecolor{mygreen}{rgb}{0,0.6,0}
\definecolor{mygray}{rgb}{0.5,0.5,0.5}
\definecolor{mymauve}{rgb}{0.58,0,0.82}
\lstset{ %
backgroundcolor=\color{white},   		% choose the background color
basicstyle=\footnotesize\monaco,       % size of fonts used for the code
columns=fullflexible,
breaklines=true,                 		% automatic line breaking only at whitespace
captionpos=b,                    		% sets the caption-position to bottom
tabsize=4,
commentstyle=\color{mygreen}\monaco,   % comment style
escapeinside={\%*}{*)},          		% if you want to add LaTeX within your code
keywordstyle=\color{blue}\monaco,      % keyword style
stringstyle=\color{mymauve}\monaco,    % string literal style
frame=single,
rulesepcolor=\color{red!20!green!20!blue!20},
% identifierstyle=\color{red},
language=python,
}
%===颜色===
\usepackage{color,xcolor}
\definecolor{dkgreen}{rgb}{0,0.6,0}
\definecolor{gray}{rgb}{0.5,0.5,0.5}
\definecolor{mauve}{rgb}{0.58,0,0.82}
 \usepackage{xcolor}
 \lstset{
  %行号
   numbers=left,
   %背景框
   framexleftmargin=8mm,
   frame=none,
   %背景色
   %backgroundcolor=\color[rgb]{1,1,0.76},
   backgroundcolor=\color[RGB]{245,245,244},
   %样式
   keywordstyle=\bf\color{blue},
   identifierstyle=\bf,
   numberstyle=\color[RGB]{0,192,192},
   commentstyle=\it\color[RGB]{0,96,96},
   stringstyle=\rmfamily\slshape\color[RGB]{128,0,0},
   %显示空格
   showstringspaces=false,
   xleftmargin=2.5em
 }

%--------------------
\hypersetup{hidelinks}
\usepackage{booktabs}  
\usepackage{shorttoc}
\usepackage{tabu,tikz}
\usepackage{float}
\usepackage{multirow}

\tabcolsep=1ex
\tabulinesep=\tabcolsep
\newlength\tikzboxwidth
\newlength\tikzboxheight
\newcommand\tikzbox[1]{%
        \settowidth\tikzboxwidth{#1}%
        \settoheight\tikzboxheight{#1}%
        \begin{tikzpicture}
        \path[use as bounding box]
                (-0.5\tikzboxwidth,-0.5\tikzboxheight)rectangle
                (0.5\tikzboxwidth,0.5\tikzboxheight);
        \node[inner sep=\tabcolsep+0.5\arrayrulewidth,line width=0.5mm,draw=black]
                at(0,0){#1};
        \end{tikzpicture}%
        }
\makeatletter
\def\hlinew#1{%
  \noalign{\ifnum0=`}\fi\hrule \@height #1 \futurelet
   \reserved@a\@xhline}
   

\usepackage{ifthen}
\newcommand{\HRule}{\rule{\linewidth}{0.5mm}}
\newcommand{\tabincell}[2]{\begin{tabular}{@{}#1@{}}#2\end{tabular}}%
%===使得公式随章节自动编号===
\makeatletter
\@addtoreset{equation}{section}
\makeatother
\renewcommand{\theequation}{\arabic{section}.\arabic{equation}}
%-------------------------
\usepackage{pythonhighlight}
\usepackage{tikz}                    
\usepackage{tikz-3dplot}
% \usepackage{hyperref}
\usetikzlibrary{shapes,arrows,positioning}
%===正文开始===
\begin{document}
%===定理类环境定义===
\newtheorem{example}{例}              	% 整体编号
\newtheorem{algorithem}{算法}	
\newtheorem{theorem}{定理}            	% 按section编号
\newtheorem{definition}{定义}
\newtheorem{axiom}{公理}
\newtheorem{property}{性质}
\newtheorem{proposition}{命题}
\newtheorem{lemma}{引理}
\newtheorem{corollary}{推论}
\newtheorem{remark}{注解}
\newtheorem{condition}{条件}
\newtheorem{conclusion}{结论}
\newtheorem{assumption}{假设}
%===重定义===
\renewcommand{\contentsname}{目录}     
\renewcommand{\abstractname}{摘要} 
\renewcommand{\refname}{参考文献}     
\renewcommand{\indexname}{索引}
\renewcommand{\figurename}{图}
\renewcommand{\tablename}{表}
\renewcommand{\appendixname}{附录}
\renewcommand{\proofname}{证明}
%\renewcommand{\algorithm}{算法} 
\renewcommand\emph[1]{\textcolor{red}{\textbf{#1}}}
%===封皮和前言===
\begin{titlepage}
\begin{center}
% Upper part of the page
\includegraphics[width=0.25\textwidth]{logo}\\[1cm]    
%\textsf{\LARGE\bfseries Natural selection, Survival of the fittest.}\\[1.0cm]
\textsf{\Large\bfseries Beijing University of Chemical Technology}\\[1.0cm]
\textsc{\Large Learn Notes}\\[0.5cm]
% Title
\HRule \\[0.8cm]
{\huge \bfseries 生产实习学习笔记}\\[0.4cm]
\HRule \\[0.7cm]
% Author
\textsf{\bfseries 计科1906 李腾飞}
\tableofcontents 
\vfill
% Bottom of the page
{创建日期:2022年7月28日}\\
{更新日期:\today}
\end{center}
\end{titlepage}
\pagestyle{plain}
\pagenumbering{Roman}
\thispagestyle{empty}
%===正文===
\pagestyle{fancy}
\pagenumbering{arabic}

%===第一章===
\section{LaTeX软件使用指导}
\subsection{软件下载和安装}
首先需要下载软件TeXLive2019(尽可能使用2019,不要装以上版本,因为我目前使用2019,没法重现错误),这个软件比较大,下载地址为:

\url{https://pan.baidu.com/s/1SXUHUEcrV6lENEALVNgH_g},提取码:ytex

下载后,进行安装,特别注意\emph{安装时间比较长,需要耐心等待},详细过程参考下面的网页:

\url{https://blog.csdn.net/weixin_45860123/article/details/111185293}

\subsection{使用TeXWorks进行编辑}
TeXWorks软件是默认的编辑器,左边为编辑区,右边为PDF文件查看器,从左上角下拉框选择XeLaTeX命令进行编译,注意需要编译两次。

\begin{figure}[H]
\small
\centering
\includegraphics[width=\textwidth]{texworks.png}
\caption{TeXWorks} \label{fig:texworks}
\end{figure}

\subsection{如何使用提供的模板}
\subsubsection{TexWorks编辑并编译}
直接双击notes.tex,使用TexWorks打开,从左上角下拉框选择XeLaTeX命令进行编译$2$次,即可生成目标pdf文件,表明安装成功。

\subsection{LaTeX常见知识}
下面就LaTeX插入数学表达式、图形、定理、表格、算法和代码提供一些例子,多练习几个例子就自动掌握使用方法了。

\subsubsection{数学表达式}
建议使用AxMath软件,生成LaTeX表达式
\begin{figure}[H]
\small
\centering
\includegraphics[width=0.8\textwidth]{axmath.png}
\caption{AxMath} \label{fig:axmath}
\end{figure}

行内公式放到\$\$内,行间公式表示如下:
\begin{align}
\begin{cases}
	\min_{\alpha} \frac{1}{2}\sum_{i=1}^n\sum_{j=1}^n\alpha_i\alpha_jl_il_jK(\bm{x}_i,\bm{x}_j) -\sum_{i=1}^n \alpha_i\\
	\begin{matrix}
	s.t.&	\sum_{i=1}^n\alpha_il_i = 0	& i=1,2,\cdots ,n \\
       &	0\leq \alpha_i\leq C	& i=1,2,\cdots ,n	 \\
\end{matrix}
\end{cases}
\end{align}

\subsubsection{定理及其证明}
\begin{theorem}\label{rotM}
平面旋转矩阵$ J $只和两向量之间的夹角$ \theta $有关
\end{theorem}
\begin{proof}
$ a_1=Rsin\alpha , b_1=Rcos\alpha~~ .~~ a_2=Rsin\beta ,b_2=Rcos\beta $
展开  $ a_2 $可以得到
\begin{align}
\label{Jc}
  {J}_c=\begin{bmatrix} cos\theta&-sin\theta\\sin\theta& cos\theta
\end{bmatrix} 
\end{align}
\end{proof}

\subsubsection{图和表}
\begin{figure}[H]
\small
\centering
\includegraphics[width=0.9\textwidth]{sample.png}
\caption{sample} \label{fig:sample}
\end{figure}

\begin{table}[H]
\caption{ 中文表}
\centering
\begin{tabular}{ll|ll}
\toprule
项目 &  取值  &项目 & 取值\\
\midrule[2pt]
$NL$ & 1/2/3   &$\theta $     & [0,1]\\
$L$ & 1000cell    &$\lambda  $    &\{0.1,0.25,0.5,10\} \\
$\tau$ &[0,1]   &${v_{limit}}$  &  10cell/time-step\\
 \multirow{2}{*}{\tabincell{c}{$\Omega  $}}  &  \multirow{2}{*}{\tabincell{c}{CCL/NCL/ACL/\\FCL}}&  $DL$  &  0/1/2/3  \\
 \\
\bottomrule
\end{tabular}
\end{table}

\subsubsection{算法描述}

\begin{algorithm}[H]
\caption{算法}\label{algorithm}
\KwIn{数据集$J(\theta_1, \theta_2, \cdots, \theta_d)$, $\varepsilon$\;}
\KwOut{全局最小点$\mathbf{\theta}^{(k)}$\;}
初始化函数参数$\mathbf{\theta}^{0}=(\theta_1 ^{(0)},\theta_2 ^{(0)},\cdots \theta_d ^{(0)})$;

$k\leftarrow 1$\;

\Repeat{$\|\mathbf{\theta}^{(k)} - \mathbf{\theta}^{(k-1)}\| \leq \varepsilon$}{
	$\theta_1^{(k)} = arg \min_{\theta_1} J(\theta_1, \theta_2^{(k-1)}, \theta_3^{(k-1)}, \cdots, \theta_d^{(k-1)})$ \;
	
	\If{cond2}{
		second if\;
	}
	
	\lIf{cond5}{cond5 true}	
	
	\eIf(\tcc*[f]{then comment}){test}{
		then with a comment\;
	}(\tcc*[f]{comment in else})
	{
		here we are in else\;
	}

 	\lFor{i=1 \emph{\KwTo} max}{mark i}
 	
	\ForEach{$e$ in the set}{
		put $e$ in ${\cal E}$\;
		mark $e$\;
	}
	
	$\theta_d^{(k)} = arg \min_{\theta_d} J(\theta_1^{(k)}, \theta_2^{(k)}, \theta_3^{(k)}, \cdots, \theta_d)$ \;
	
	$k\leftarrow k + 1$\;
}
  
\Return $\mathbf{\theta}^{(k)}$ \;
\end{algorithm} 


\subsubsection{代码描述}
\begin{lstlisting}[language={c++},
        numbers=left,
        numberstyle=\tiny\monaco,
        basicstyle=\footnotesize\monaco]
#include<iostream>
using namespace std;
 
void print(int arr[], int n)
{  
    for(int j= 0; j<n; j++)
	{  
           cout<<arr[j] <<"  ";  
        }  
    cout<<endl;  
}  
 
void BubbleSort(int arr[], int n)
{
    for (int i = 0; i < n - 1; i++)
	{
            for (int j = 0; j < n - i - 1; j++)
	        {
                    if (arr[j] > arr[j + 1]) 
			{
                            int temp = arr[j];
                            arr[j] = arr[j + 1];
                            arr[j + 1] = temp;
                        }
                 }
         }
}
 
int main()
{  
    int s[10] = {8,1,9,7,2,4,5,6,10,3};  
    cout<<"初始序列:";  
    print(s,10);  
    BubbleSort(s,10);  
    cout<<"排序结果:";  
    print(s,10);  
    system("pause"); 
}
\end{lstlisting}


\subsection{编译出现The font Monaco cannot be found错误,如何处理?}
是因为缺乏字体。字体文件下载地址如下,提取码为:rrqu 

\url{https://pan.baidu.com/s/14_i7Xova3ozFSVeoGXldRg?pwd=rrqu }

Win10系统中在超级管理员权限下执行如下命令(Monaco.ttf同目录):
\begin{lstlisting}
copy Monaco.ttf C:\Windows\Fonts
\end{lstlisting}

Win10系统中在超级管理员权限下执行如下命令(Monaco.ttf同目录):
\begin{lstlisting}
copy Monaco.ttf C:\Windows\Fonts
\end{lstlisting}


\subsection{总结}
LaTeX是一个非常有用的排版软件,希望大家学好用好。



%===参考文献===
%\addcontentsline{toc}{section}{参考文献}
%\bibliographystyle{abbrv}     %论文引用格式
%\bibliography{E:/studio/wrtex/wrtkit/referbib/wholebiblio}
                         
\begin{thebibliography}{99}
\bibitem{A19}
{\em \color{red}latexstudio.net}. \url{http://www.latexstudio.net}, 2021.
\end{thebibliography}
\end{document}
%===结束===



History:
2022-8-20: 依据程勇老师的Note.tex创建;



